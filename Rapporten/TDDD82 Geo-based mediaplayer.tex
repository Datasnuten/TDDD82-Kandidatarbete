\documentclass[9pt,a4paper]{acmproc}
\usepackage[utf8]{inputenc}
\usepackage{amsmath}
\usepackage{amsfonts}
\usepackage{amssymb}
\usepackage{graphicx}
\usepackage[english]{babel}
\usepackage[utf8]{inputenc}
\usepackage{amsmath}
\usepackage{cite}

\graphicspath{}

\author{
\texttt{Andreas Nordberg}\\
\texttt{andno793@student.liu.se}
  \and
  \texttt{Jonathan Sjölund}\\
  \texttt{jonsj507@student.liu.se}
}

\begin{document}
\pagenumbering{gobble}

\title{%
	Geo-based mediaplayer \\
	\large An interactive interface for geo-based video streaming}
\maketitle

\begin{abstract}
Being able to interact with a video stream can be both fun and educational; however for it to accomplish that the interaction must be seamless. Creating a media player that can allow user to view a video stream from different geographical position is a challenge that is tackled in this paper. By providing detailed information of the material used and method this paper will provide the information necessary the possibilities of that kind of media plater and explain how to achieve a seemingly good and to a high extent an enjoyable video streaming experience. A proof-of-concept is shown and provides a result that...(Mention the result here)

\end{abstract}

\begin{keywords}
HTTP Adaptive Streaming, OSMF video player, Interactive video streaming, Geographical based streaming
\end{keywords}

\section{Introduction} 

\subsection{Motivation}
Streaming has evolved and become more popular over last couple of years. Thousands upon thousands of different streams are being watched and streamed by many people every day, thus the demand for better and more ways to stream are longed for. Stream videos in different ways in order to create a more interesting streaming environment that can provide both better entertainment but also a better way to improve observation in science and other areas more reliably. If a stream can provide the possibility for watching a video from different angles it can give people the option to observe and also enjoy something from different perspectives. This project focuses on accomplishing that, by creating a geo-based video player that can allow people to view a video from different angles and change between them seamlessly without any delay or stuttering. By looking at an existing video streaming player and improve it to accomplish this task we can show and help to prove that it is something worth implementing in already existing video players.

\subsection{Narrow Scope}
In this project a so called Geo-based Command-and-control video streaming player using geo-tags will be designed and developed. In practice this means a service in which you can choose between a set of recording streams of, for example, the same event but slightly different locations and angles. This would be a useful feature to have in any larger event where you would want to show the same scene from different locations and angles. The interface will be useful for both event-organizers who hires staff to make several different recordings of the same scene for simultaneous viewing, but could also be used by the public who volunteers to record the event.

\subsection{Methodology}
To advance in this project we will mainly be programming, designing and developing the app. During the project we will receive constructive feedback about our work to help us with our process and provide us with other point of views. For any obstacles, expected and unexpected, we will read up on relevant theories that might help us.

\subsection{Boundaries}
The application we provide is only going to be a prove of concept which means that we will only focus on the functionality of the video player. Factors like a pretty interface and usability on broader spectrum will be neglected. We will focus on making the application work for only one user to verify the functionality we want to accomplish. The number of video streams that we will be able to seamlessly switch between will be limited to only three to four. This is also because we only want to verify that it is possible to switch between video streams and not that it can be done over large numbers. Reason for this is that pre-buffering many videos can be hard to accomplish on large number of video streams, as long as we provide a way to do it the solution can be expanded upon.

\clearpage

\section{Background}

\subsection{Theory}

\subsection{Method and material}

\clearpage

\section{Discussion}

\subsection{Result}

\subsection{Conclusion}

\clearpage
Ignore the references for now, it is only for us to see how to write references.
\begin{thebibliography}{9}

\bibitem{impMult}
  Jim Geier,
  \emph{Implementing Wi-Fi Multicast Solutions}.
  \newline
  Published: 2004-11-09. Fetched: 2015-09-25 \newline
  URL: http://www.wi-fiplanet.com/tutorials/article.php/3433451/Implementing-Wi-Fi-Multicast-Solutions.htm
  
\bibitem{briefHist}
	Beau Williamson,
	\emph{Developing IP Multicast Networks}.
	\newline
Cisco Press; 1 edition (October 29, 1999).

\bibitem{multExplained}
  Juan-Mariano de Goyeneche,
  \emph{Multicast Explained}.
  \newline
  Published: 1998. Fetched: 2015-09-25 \newline
  URL: http://www.tldp.org/HOWTO/Multicast-HOWTO-2
  
\bibitem{whatIsMult}
  Windows Server,
  \emph{What Is IPv4 multicasting?}.
  \newline
  Fetched: 2015-09-25 \newline
  URL: https://technet.microsoft.com/en-us/library/cc772041(v=ws.10).aspx
  
  \bibitem{understandIpMult}
  Page Administrator,
  \emph{Multicast - Understand how IP multicast works}.
  \newline
  Fetched: 2015-09-25 \newline
  URL: http://www.firewall.cx/networking-topics/general-networking/107-network-multicast.html
  
  \bibitem{udpSource}
  J. Postel,
  \emph{User Datagram Protocol}.
  \newline
  Published: 1980. Fetched: 2015-10-15 \newline
  URL: https://tools.ietf.org/html/rfc768
  
   \bibitem{graphSource}
  Cisco Systems, Inc.,
  \emph{Multicast Deployment Made Easy}.
  \newline
  Published: 1999. Fetched: 2015-10-15 \newline
  URL: http://ftp.icm.edu.pl/packages/cisco-ipmulticast/whitepapers/Multicast-Deployment-Made-Easy.pdf
  
%\bibitem{randall2008}
%	Randall D. Knight,
%	\emph{Physics for Scientists and Engineers: A %Strategic Approach with Modern Physics (2nd %Edition)}.
%	\newline
%Addison-Wesley, 2a upplagan, 2008.

%\bibitem{polesAndZeros}

%	\emph{Understand Poles and Zeros}.
%	\newline
%	Publicerad 2002-11-01. Hämtat: 2015-06-01.
%	\newline
%	URL: http://web.mit.edu/2.14/www/Handouts/PoleZero.pdf

\end{thebibliography}

\end{document}
