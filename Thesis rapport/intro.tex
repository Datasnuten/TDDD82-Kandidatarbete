\chapter{Introduction}
\label{cha:introduction}

Streaming has evolved and become extremely popular over the last decade. Millions upon millions of different streams are being watched every day\footnote{Twitch statistics https://stats.twitchapps.com/, Fetched: 2016-04-01}. Thus, the demand for better and more ways to stream and view streams are longed for. If we could stream videos in different ways we can create a more interesting streaming environment. If a stream can provide the possibility for watching a video from different angles it can give people the option to observe and also enjoy something from different perspectives. Carlsson et al.\cite{optimizedstreaming} have considered optimized prefetching projects for this context. This project complements this, by creating a geo-based video player that uses HTTP-adaptive streaming (HAS) which allows for users to view a video from different angles and change between them seamlessly without any buffering delay or stuttering. By extending the functionality of an existing video streaming player and generalizing it to offer this service, we demonstrate that it is possible and worthwhile to implement this feature in already existing media players.

In this project, we design and develop a geo-based command-and-control video streaming player using geo-tags. In practice, this is a service in which you can choose between a set of recording streams of the same event, for example, but slightly different locations and angles. This would be a useful feature to have for any large event where you would want to show the same scene from different locations and angles. For easy user interaction the interface should then be able to automatically incorporate the coordinates from which these streams where recorded and display their relative geographic locations on the user interface. The interface will be useful for both event-organizers that hire staff to make several different recordings of the same scene for on-demand viewing, but could also be used by the public who volunteer to record the event live. One major thing this interface also could be used for is during a disaster event or something of the sort. For example, such interface could help the police, medical or the emergency service to view a disaster scenario from multiple angles, helping them understand the situation and help then in their communication. In such a scenario, being able to swap between different video streams would give them a better understanding of the scenario and what needs to be done in their work.

\section{Boundaries}
\label{sec:boundaries}

The application we provide is only going to be a proof-of-concept, which means we will only focus on the functionality of the video player. Factors like a designing a pretty interface and a more extensive focus on user friendliness on broader spectrum will have a low priority. We will focus on making the application work for one user to verify the functionality we want to accomplish. The number of video streams that we will initially be able to switch between will, for the purpose of testing, be limited to a few but then expanded upon to support any reasonable number of streams. This is because our main focus is to make sure that it is possible to switch between video streams, not that it is possible to do so with a large number of streams. The reason for this is that pre-buffering many videos can be difficult to accomplish with a large number of video streams and it can come with a tradeoff of bandwidth usage \cite{watchingprefetching} and less efficient bandwidth usage when downloading in parallel \cite{scalableOnDemand}. As long as we provide a way to make it function for a few streams the solution can be expanded upon afterwards.