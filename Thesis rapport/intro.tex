\chapter{Introduction}
\label{cha:introduction}

Streaming has evolved and become more popular over the last couple of years. Thousands upon thousands of different streams are being watched every day\footnote{Twitch statistics https://stats.twitchapps.com/, Fetched: 2016-04-01}, thus the demand for better and more ways to stream and view streams are longed for. If we could stream videos in different ways we can create a more interesting streaming environment, this can provide both better entertainment but also a better way to potentially improve observation in science and other areas more reliably. If a stream can provide the possibility for watching a video from different angles it can give people the option to observe and also enjoy something from different perspectives. This project focuses on accomplishing this, by creating a geo-based video player that uses HTTP-adaptive streaming (HAS) that can allow people to view a video from different angles and change between them seamlessly without any buffering delay or stuttering. By looking at an existing video streaming player and improving it to accomplish this task, we show that it's a feature worth implementing in already existing media players.

In this project we design and develop a geo-based command-and-control video streaming player using geo-tags. In practice this is a service in which you can choose between a set of recording streams of the same event, for example, but slightly different locations and angles. This would be a useful feature to have in any larger event where you would want to show the same scene from different locations and angles. The interface should then be able to automatically accept multiple coordinates for these streams and draw them in this graphical interface. The interface will be useful for both event-organizers that hire staff to make several different recordings of the same scene for simultaneous viewing, but could also be used by the public who volunteer to record the event live. One major thing this interface also could be used for is during a disaster event or something of the sort, by helping the police, medical or the emergency service by allowing them to view a disaster scenario from multiple angles to help them in their communication. In such a scenario, being able to swap between different video streams would give them a better understanding of the scenario and what needs to be done in their work.

\section{Boundaries}
\label{sec:boundaries}

The application we provide is only going to be a proof-of-concept, which means that we will only focus on the functionality of the video player. Factors like a designing a pretty interface and a more extensive focus on user friendliness on broader spectrum will be neglected. We will focus on making the application work for only one user to verify the functionality we want to accomplish. The number of video streams that we will initially be able to switch between will, for the purpose of testing, be limited to a few but then expanded upon to support any reasonable number of streams. This is because our main focus is to make sure that it is possible to switch between video streams, not that it's possible to do so with a large number of streams. The reason for this is that pre-buffering many videos can be difficult to accomplish with a large number of video streams and it can come with a trade-off of bandwidth usage \cite{watchingprefetching} and less efficient bandwidth when in downloading in parallel \cite{scalableOnDemand}. As long as we provide a way to do it the solution can be expanded upon afterwards.