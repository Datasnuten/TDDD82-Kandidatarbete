Being able to interact with video streams can be both fun, educational and provide help during disaster situations. However, to achieve the best user experience the interaction must be seamless. This thesis presents the design and implementation of an interface for a media player that allows for users to view multiple video streams of the same event from different geographical positions and angles. The thesis first describes the system design and methods used to implement this kind of media player and explains how to achieve a seemingly good and, to a higher extent, enjoyable video streaming experience. Second, an algorithm is developed for placing each video stream object on the interface's geographic-based map automatically. These objects are placed to ensure the relative positions of the objects compared to the real world. The end result of this project is a proof-of-concept media player which enables a user to see an overview over a geographical streaming area. Presented with the relative location of each stream to the point of interest the player allows the user to click on that stream and switch to viewing the recordings from that point of view. While the resulting player is not yet seamless, the result of this project shows the command-and-control center as initially envisioned. Implementing seamless, uninterrupted, switching between the video streams is outside the scope of this thesis. However, as demonstrated and argued in the thesis, the work done here and the developed software code will allow for easy integration of more advanced prefetching algorithms in future and parallel works.