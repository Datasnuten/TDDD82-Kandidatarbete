Being able to interact with video streams can be both fun, educational and provide help during disaster situations however, to achieve the best user experience, the interaction must be seamless. Creating an interface for a media player which allow for users to view multiple video stream of the same event from different geographical positions and angels is a challenge that is tackled in this paper. This paper describes the material and methods used to accomplish this kind of media player. And it explains how to achieve a seemingly good and, to higher extent, enjoyable video streaming experience. An algorithm is developed for placing each video stream object on the interface's geographical based map automatically. These objects are placed to ensure the relativite positions of the objects compared to the real world. The end result of this project is a proof-of-concept media player which enables a user to see an overview over a geographical streaming area. By seeing each streams relative location to the point-of-interest the users are able to click on that stream and switch to it, however not in a seamless way. The result of this project shows the command-and-control center as initially envisioned but without seamless, uninterrupted switching between the video streams. Instead the work done and the end result code provided will allow for prefetching to be easily patched in into any future work.