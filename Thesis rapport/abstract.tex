\texttt{Being able to interact with a video stream can be both fun, educational and provide help during disaster situations however, to achieve the best user experience, the interaction must be seamless. Creating a media player that can allow users to view a video stream from different geographical positions is a challenge that is tackled in this paper. This paper describes the material and methods used to accomplish this kind of media player. And it explains how to achieve a seemingly good and, to higher extent, enjoyable video streaming experience. An algorithm is developed for placing each object on a geographical based map automatically. The objects are placed to ensure that relativity compared to the real world is kept. A proof-of-concept is shown of a media player that enables a user to see an overview over a geographical streaming location. By seeing each streams relative location to the point-of-interest the users are able to click on that stream and switch to it, however not in a seamless way. This project shows the command-and-control center envisioned but without a seamless switch. Instead the work done and the code provided will allow for prefetching to be added in future work in an easy way.}