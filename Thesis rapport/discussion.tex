\chapter{Discussion}
\label{cha:discussion}


During the project we faced a lot of obstacles and some things which needed to be changed. In the sections below these problems will be described, why they may have happend and how they could be fixed. What changes were made and had to be done will be discussed and also what could have been done differently.


\section{Understanding the provided code}
\label{sec:understandingcode}

When we started working on this assignment to make an interactive Command-and-Control center with geo-tagged streaming we first had to install and adjust to the tools given to us to develop the interface, being OSMF and SMP. These tools consisted of an extensive amount of existing code which we had to delve into and understand for us to implement our features.
This was a process which took some time since we also weren’t very familiar with the language enviroment, Adobe ActionScript 3.0. ActionScript is an object-oriented programming language developed by Adobe Systems and influenced by JavaScript, while its syntax still being relatively similar to Java which we had previous experience with. Eventually we got a better understanding on how to operate in this new enviroment and reverse engineer the provided code. However there were still many sections of the code which we didn’t understand or knew that we would need in our work, and wrapping our heads around this took more time than we initially expected. 

\section{Issues with HAS and prefetching}
\label{sec:hasissues}

At the start of this project we focused and spent much of our time on understanding the principles of HAS, geographical based streaming, prefetching and how to implement them into our own interface. While we did have a good grasp on how these principles works and had a good idea of how we would go around to implement them, we couldn’t quite get it to work. Since we used code from a previous work we made the assumption that as long as our implementation of our interface’s features was similar to that previous work, the HAS would function. Flash builder, SMP and the HAS-functionality in the provided code required the video files to be split into the formats F4M, F4X and F4F when doing the prefetching. We were also provided with some video test files from our supervisor which he had successfully used when he worked on the HAS-functionality in his code. This however didn’t work for us since some codebits didn’t run properly. There are two things that may be the cause of this. The first thing is that we didn’t do what was neccessary to get it to work because our lack of understanding of how the HAS-functionality actually operates in the code and how we would need to rewrite the existing code to function with swapping between several videos. It didn’t work out of the box because HAS in the provided code was hard coded to only support one video and our attempts at supporting multiple video streams ended in failure even with the assistance of the HAS-functionality code’s author himself. The second cause of this might be because the changes we did to the provided code in our implementation ruined the functionality of HAS. If we were to look at those two cases the first one seems to be the more plausible one, since we assumed that the code we got would just work as long as we had the assets and did a similar implementation to the one our supervisor had done. The second one seem less likely since the changes we made to the code was so that it wouldn’t disrupt the HAS or media player in anyway, however it could also be a possibility. 

Because we couldn’t get the HAS-functionality to work properly we therefore couldn’t get the prefetching of different video streams to work. Our focus and time throughout most of the project was very much put on the prefetching, but since we couldn’t get it to work we switched our focus to a better implemented and functional command-and control interface. This included improving the interface to work properly whether the player was in standard or fullscreen mode, each geographical map object displaying GPS-coordinates and direction of the video stream while hovering over it and the relative position placement algorithm for drawing the objects. The position algorithm took some time to implement but we had initially a general idea of how it should work. When we developed it we worked on two similar but separate solutions each to see which one worked best, but since it took more time than expected only one solution was finished in time which proved feasible and then used. The main challenge with developing this algorithm was to provide relativity, scalability and accuracy up to our standards which caused the algorithm to take some time to create.

\section{Improvements the position algorithm}
\label{sec:posimp}

When developing the position algorithm we looked at several ways to translate the spherical longitude and latitude to accurate grid x- and y-coordinates. In the end the choice was made between the two formulas haversine and equirectangular approximation. The formula we decided to use in the end was equirectangular projection because that’s the first one we tried to implement with the algorithm and it worked well. Since the accuracy of equirectangular approximation apparently is slightly worse than that of the haversine formula, we could have compared the use of both formulas to see if there were any significant difference in the implementation between the two. Nonetheless the final algorithm is up to the standard that we envisioned. 

\section{The test case}
\label{sec:test case}

For our test case, there is one thing we in hindsight would have changed if we would have redone it. In our case we set up only two cameras at a time to get multiple views of what was happening at the scene from different locations, simultaneously. To further and better prove the functionality of our user interface in a test case, we should have brought some more volunteers and cameras along with us to get even more point of views of the same scene at one point in time. While doing two recordings at once was enough to prove the functionality of this feature, more recordings would have been a better addition. 

\section{Adobe Flash}
\label{sec:adobe flash}

Furthermore, as mentioned previously in this report, Adobe Flash is becoming more deprecated by the day even by Adobe themselves. Because of this, if the project was redone the interface would be better suited to be implemented in the media player built from a more modern alternative such as Flash’s main competitor, or rather replacement, HTML5.

\section{Issues with the server}
\label{sec:serverissues}

One big obstacle which unnecessarily cost a lot of time was setting up the server we used. Initially we used something called a WAMP\footnote{WAMPSERVER: http://www.wampserver.com/en/} server at the start of the project which enabled us to stream videos using HTTP through an Apache HTTP Server. However, since idea of prefetching was still present at that point of the project there was a need to switch to Adobe Media Server 5 since it would allow us to stream chunked bits of video used for the prefetching. While setting up the servers we ran across numerous problems with different kinds of security errors which wouldn’t allow us to stream the videos using HTTP. While trying to solve these issues we found that since the Apache server ran on a Windows 10 client there was a process that blocked the server that was needed to be stopped\footnote{For Windows 10 use the following command to stop the process blocking Apache: iisreset /stop}. Only then was the server able to run and allow videos to be streamed with HTTP.

\section{Project structure improvements}
\label{sec:psi}

If the project was redone we would have made a more definite time plan of what was needed to be done. Our time plan, even though straigthforward, wasn’t very detailed. We knew what we wanted to accomplish and when but we didn’t really know how we would go about to accomplish it. This ended up unnecessarily consuming a lot of time since we didn’t know where to look in the giant web of provided code to solve any eventual issues or where exactly to implement the changes and solutions.