\chapter{Conclusion}
\label{cha:conclusion}

This project provides a command-and-control center UI which allows for video streams to be changed on-demand with the use of an interactive geographical position map, with video stream locations tagged with GPS-coordinates including latitude and longitude values and cardinal directions, implemented in Strobe Media Playback. The interactive map provides details of where streamers are positioned relative to each other. This is handled through an algorithm which places every object on the map relative to how the objects, representing the recordings, in the real world are located. The accuracy of the algorithm is shown to be placing each object relatively good to one another with a good accuracy, at least when the number of objects does not exceed ten. By creating an object with a latitude, longitude and direction the interface will show this information of the object while hovering over it. Besides the locations of the streams there is also a point of interest drawn in the map which is also using this same algorithm. Each stream is clickable and displays its representative video in the media player when clicked. The object representing the currently played recording will then be highlighted when the geographical map is reopened to show that this video is currently loaded and played. All the videos are interactable and when switching videos the point of time in the recording will be transferred to the next video in order to allow for the user to see the same situation and swap between different locations and angles. A test was done to see how long it would take for a video to load when clicking between different streams. The result shows that the average time is about a seventh of second which seems almost instantaneous. Even though seamless, uninterrupted switching through prefetching was not achieved because of several difficulties, the final code of this project is made in a way that prefetching should be easy to implement with our developed interface. The features of this interface should allow for a good way for people to stream and interact with videos during a concert or disaster event in way that will make experience and work easier to accomplish. 

For future work, when prefetching is added, the switching between different videos will be seamless and further improve the quality of service of the geo-based media player.